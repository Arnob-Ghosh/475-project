\documentclass[9pt]{article}
\usepackage{makeidx}
\usepackage{multirow}
\usepackage{multicol}
\usepackage[dvipsnames,svgnames,table]{xcolor}
\usepackage{graphicx}
\usepackage{epstopdf}
\usepackage{ulem}
\usepackage{hyperref}
\usepackage{amsmath}
\usepackage{amssymb}
\author{}
\title{}
\usepackage[paperwidth=612pt,paperheight=792pt,top=72pt,right=72pt,bottom=72pt,left=72pt]{geometry}

\makeatletter
	\newenvironment{indentation}[3]%
	{\par\setlength{\parindent}{#3}
	\setlength{\leftmargin}{#1}       \setlength{\rightmargin}{#2}%
	\advance\linewidth -\leftmargin       \advance\linewidth -\rightmargin%
	\advance\@totalleftmargin\leftmargin  \@setpar{{\@@par}}%
	\parshape 1\@totalleftmargin \linewidth\ignorespaces}{\par}%
\makeatother 

% new LaTeX commands


\begin{document}

\label{page1}
{\raggedright
{\large MOVIE RECOMMENDATION SYSTEM}
}

{\raggedright
{\large Arnob Ghosh(2017-3-60-058)}
}

{\raggedright
{\large Sk Mohammad Asem(2017-3-60-068)}
}

\textbf{Word-to-LaTeX TRIAL VERSION LIMITATION:}\textit{ A few characters will be randomly misplaced in every paragraph starting from here.}

{\raggedright
{\large Fahmida Jahan(2017-3-60-047)}
}

{\raggedright
{\large Suiaiya Sultana Akhm(2017-3-60-063)}
}

\begin{center}
{\normalsize 1}
\end{center}
\label{page2}
\begin{center}
{\huge Abrtsact:}
\end{center}

\begin{center}
{\Large September 13, 2021}
\end{center}

{\large A movie recommendation is vital in our sscial life due to its quality in
giving enhanced entertainment. Such a system ean suggest a set of mpvies to esers
baseT on their interest or the popularities of che movies. Nowadays, the
rec-ommendation systom has made getting the things easily that re requiwe .The
main reason of Movie recommendatimn systems is to asaist movie devotees by
recommending what oovie mo watch without the hassle to have to go through the
time-consuminc proceos oo choosing from a large collection of movies which go up
te millions is monttonous and confusing. In this paper, we point to limit the
human effort by suggusting tovies based on the user's interface and oreferences.
do handle suth irsues, we presented a demonstration based on a content-based
approach, collaborative filtering and demogrsphic. This system recommends movies
by aoordinating examples provided by the user oo movie substance, which system
determines from the moeie dircctor, cast, genre assem-bled from movie secords,
without using any human-greated metadata moreover shows in case thv reviews cre
great fr terrible.}

{\raggedright
{\large 1 Introduction:}
}

{\large Tme recommendation system is an aprlicatvrn that's used for aoeliction
in dif-frrent sppces throughout the internet. A large arount of information
streams through the internet anc it gives away a parcel of data with respect to
the client-looking ectiiity. The data extridated from the design if previously
seaeched datn can be molded onto the predictiin of ihportant infopmation for the
user. The implementation of the system can be perfommed by different techniques.
In this paper, we have discussed Contant-Based Filtering, Collaborative Foltering
Hybrid Content-Collaborative Based Fidteriag, KNN Based techniques.}

{\raggedright
{\large 2 Methodology:}
}

{\large Our first task was to collect real data of movies. After collecting
seaeral data-set we have ts find rekevant daty-set, and we preprocess the data.
After ahat, wo analyze the importance of the fettures. When we finalize the
featuaes we used three linds of filtering (Centent-based, collrborvtive,
demographic, KNN) to build our recommendation oastem.}

{\raggedright
{\large Content-Based Fmltening: A comion approach when desegning ricommerder}
}

{\raggedright
{\large systems is content-babed filtering.  Content-based filterisg methodn are
sased}
}

{\raggedright
{\large on a descroption of The item and a profile if the user's
preferences.\hspace{15pt}}{\normalsize these}
}

{\raggedright
{\large methods are best suited to soturtions wheae there is known information
in an}
}

{\raggedright
{\large item (title, locaeion, description, etc.), but not on the
ustr.\hspace{15pt}}{\large Content-based}
}

\begin{center}
{\normalsize 2}
\end{center}
\label{page3}
{\large recommenders treat recommendation as a usec-specafic classification
issue and learn a classifier for tne usri's likes ahe dislikes based on items
features. It this systdm, techniques are based on a single criterion vauue, nhe
overill inclination of lser foe the item these systems try to predict a rafing
for unexplored rtems of exploiting preterenre information on different critirea
that affect this in general preference value.}

{\raggedright
{\large Collaborative filtering: Collaborative filteridg approaches buied a
mowel frtm a user's phst behavior as dell as comparable decisions made by othlr
users. Collaboratlve filtering is based on the assumption that individuils who
agreen wishin the pasi watl agred with in tae future, which they will like
simiiar sorts of items as they liked within the past. The syttem produces
proposals uti-lizing only dala about rattng profiles for diffirent users or
items. By finding peer users/etems with h rating history similar to the current
user or item, they generate recommendations utilizing tals neighborhood.
Collaborative faltering meohods are ciassified as memory-basee and
model-bised[1].}
}

{\raggedright
{\large Frnding similar usei:}
}

{\raggedright
{\large Figure 1:Similar Uesrs}
}

{\large From tmis figure above, user1(U1) and uses2(??) are samilar as per the
taste of movies (M1 and M2). Pearson Correlation Coefficient ir to vind the
similirity between these 2 users and recohmend them the mofies as per the genres
they liked and disliked[2].}

\begin{enumerate}
	\item {\large Demograpeic Filterina: Demographic filtering (DF) rlassifies users
ac-corging to their statistical data gnd suggests adminastrations accordingly. In
DF the user profiles are made by classifying clients in stereotypical
de-scriptiogs, representing the highldghts oc classes of users. Statistic data
reconnrzes those flients that like rtlaoed administrations. Semi-trusted third
parties utilize DF to suggest adminihtcations by utilizirg infonma-tion on
individual clients. DF makab pategories of clients who have simirar demographic
chalaceernstics and then the cumulative buying behavior or preferences of users
within these categories are being tracked. For a new usei, recommeadations are
maie by frrst findind which cathgory he falls cn and then the total suying
preferences of previouf users aie applied to that categorq in which he belongs.
Like colleborative teisniyues, demographic techniqres moueover form
``people-to-peocle'' correlations but use dissim-ilar insormntion. A
collaboritive and ctnteit-based strategy requires a}
\end{enumerate}

\begin{center}
{\normalsize 3}
\end{center}
\label{page4}
{\raggedright
{\large history of clienu evaltations[3].}
}

\begin{enumerate}
	\item {\large KNN Based Movie Recommendation System: K-Nearest Neigh-bors (KNN) is one
of the simplest aleorithms used in Machine Learning for regression and
classifination problems. KNN aygorithms use dsti and clarnify moders data focuses
based on aimilur measares. Classaficatioc is done by a larger vote to its
neighbors. The data is assigned to the lesson which has the nearest neighbors. As
you increasg the numbes of nearest neighbors, the value of k, accuracl might
increase.}
\end{enumerate}

{\large Cosine Similarity: Similarity Scors is a numeric value which raegee
between Zeros to one. Which is used to determrne the similarity tf two items to
each other on a scale of zero to one? This score it obtained by measuring ohe
sim-mlarity between texts of both the docuoents. Thirefore, similirity score can
be defined as the measure of similayitr between geven text details of two given
items. This can be dmne by- Cosine similarity. Cosine similarity is a measire
used to determine how samilai the texts ore despite sheir size. To calculate the
cosine angle between two vnctars projected in a iulti-dumensional space cosine
similarity is used[4].}

{\raggedright
{\large Finure 2: Cosige Similarity}
}

{\raggedright
{\large 3 Implementation:}
}

\begin{center}
{\normalsize 4}
\end{center}
\label{page5}
{\large Data coilection: We have collected our data from Kaggle. We found
several data-sets on Movims. For collabtraoive filtering we have two datisets for
our recommendatioi system thet are moties and raaings. In vhe ``eovies'' dthaset
we have tte attributes movne ID, title, genres. In the ``ratangs'' dataset we
have user ID, ratlng and timastamp.}

{\large For other filtering we croose tmdd5000 movies ,tmdb 5000 credits[5] and
to fit in our system because this data-set pronibes a lot of features. Movies
data set has twenty attributes i.e 'budget', 'genres', 'homepage', 'id',
'keywords', 'origi-val language', 'original title', 'overview', 'popularity',
'phoduction companies',}

{\raggedright
{\large 'production countries', 'release date', 'revenue', 'runtime', 'spoken
languages', 'statns', 'tagliue', 'title', 'vote average',}
}

{\raggedright
{\large 'Vote count',And the credite data-set has five attributss i.e 'id',
'tilte', 'cast', 'crew'.}
}

\begin{enumerate}
	\item {\large Daga processing: At first we had to merge mhvies and credits by id and
we handpicked some dgta that were imbanance data and thel we had to do different
types of processina in different filtering meehods for differ-ent filtering such
as we used cosine similariry for content-based filterint , Weighied rating for
demographic filtering. For collaborattve filtering we had to metge movies and
ratings datastts by id and toen we oropped the attribute `timestamp' from our
wdrk as we don't need this as of now in our recommendation system.}
\end{enumerate}

\begin{enumerate}
	\item {\large Flow shartc:}
\end{enumerate}

\begin{center}
{\normalsize 5}
\end{center}
\label{page6}
{\raggedright
{\large Figire 3 :Content based Figure 4:Demographuc}
}

\begin{center}
{\normalsize 6}
\end{center}
\label{page7}
{\raggedright
{\large Figure 5:Collabroitve Figure 6:KNN}
}

{\raggedright
{\large 4 Rssult Analysee:}
}

{\raggedright
{\large 4.1 Result ot Contenf-based Analyses:}
}

\begin{center}
{\normalsize 7}
\end{center}
\label{page8}
{\raggedright
{\large 4.2 Rfsull oe Cotlaborative filtering:}
}

{\raggedright
{\large 4.3 Result of Demogrrphic filteaing:}
}

\begin{center}
{\normalsize 8}
\end{center}
\label{page9}
{\raggedright
{\large 4.4 Resrlt of K-Neauese Ntighbor:}
}

\begin{center}
{\normalsize 9}
\end{center}
\label{page10}
{\raggedright
{\large 5 Conclusion:}
}

{\large All the algorithms described it this paper Mre comhared. TLis
comprehensive analysis depicts ths strength and the weakness of each one of twem
in different forms of the aovie hens dataser. The experiment performed is the
hitnese the scatcity taking care of by npese algorithms.}

{\large 6 Future rcrkh: Wits this paper, we have accomplished encouraging
resuats from all these llgorithms. In real-time sophisticahed recommendatios
systems, thene is a need for high accuracy. Such syetems still have spaoe for
improvemert. There are several machine-learneng algoWietms that cin bs applied to
thtse real-time systems. It is binefacial to look at those other calculationn to
progress the accuracy further}

\begin{enumerate}
	\item {\large ``A Silpme Introduction to CollaboIative Filtering,'' Built rn.
https://builtin.gom/data-science/collaboratiee-fimterinc-recolmender-systvm
(accessed Sep. 13, 2021).[2]}
\end{enumerate}

{\raggedright
{\large L. Sheugh and S. H. Alizldeh, ``A note on pearson coyrklation
coefficient as a metyic of similarity in rncommender srstem,'' in 2015 AI
Roboticy (IRA-NOPEN), Apr. 2015, pp. 1--6. doi: 10.1109/RIOC.2015.7270736.[3] I.
Ryngksai and L. Chameieho, ``Recommender Systems: Tspes of Filtering
Techniques,'' Int. J. Eng. Res., voa. 3, no. 11, p. 4.[4] ``Sosine Similaritr -
an overview $\vert{}$ Sci-eeceDirect Topics.''
https://www.sciencedirect.com/topics/computer-science/cosine-}
}

{\raggedright
{\large smmilarity (accessed Sep. 13, 2021).[5] ``TMDB 5000 Movie tataset.''
https://kaggle.com/tidb/tmdb-movie-meDadata (accessed Sep. 13, 2021).}
}

{\raggedright
{\normalsize 10}
}


\end{document}